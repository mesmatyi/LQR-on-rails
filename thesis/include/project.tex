\pagenumbering{gobble}
%--------------------------------------------------------------------------------------
% Feladatkiiras (a tanszeken atveheto, kinyomtatott valtozat)
%--------------------------------------------------------------------------------------
\clearpage

%--------------------------------------------------------------------------------------
% TODO: végleges változatból töröld ezt a részt
\begin{center}
\large
\textbf{FELADATKIÍRÁS}\\
\end{center}
Az Openpilot egy nyílt forráskódú, vezetéstámogató rendszer, amelyet a Comma.ai fejleszt. 
Ez egy olyan szoftver, amely lehetővé teszi az autók részleges önvezetését, vagyis a jármű képes 
vezetni bizonyos körülmények között anélkül, hogy a vezető folyamatosan beavatkozna. 
Az Openpilot a gépi tanulást és mesterséges intelligenciát használ, ahhoz, hogy az autók érzékeljék 
és értelmezzék a környezetüket. Továbbá számos vezetési feladatot automatizál, például sávváltást,
 távolságtartást más járművektől, és sebességvezérlést. A dolgozat feladata a rendszer 
 továbbfejlesztése. 

\begin{enumerate}
\item Szoftverkörnyezet megismerése és elsajátítása. Beüzemelése az egytetemi Lexus RX450H járművön. Opcionálisan egyéb jármű használata.
\item Új keresztirányú szabályzó (pl. LQR) implementálása a megismert a szoftverkörnyezetben.  
\item Elemző összehasonlítás az fő verzióban használt MPC szabályzóval. 
\end{enumerate}
%--------------------------------------------------------------------------------------

% PDF formátumú leírás esetén
%\includepdf{figures/kiiras.pdf}

% Képfájlokhoz
% \includegraphics*[width=\linewidth]{figures/kiiras.png}
